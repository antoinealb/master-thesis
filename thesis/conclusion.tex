\chapter{Conclusion}

We presented an extension to the \gls{r2p2} transport protocol that offers reliable, in-order, delivery of messages to application.
Our implementation uses well-known techniques like kernel bypass to offer both low tail latencies and high throughput.
We validated our design by running it against a series of microbenchmarks simulating different application workloads.
We observe an additional tail latency of only \SI{13}{\micro\second} compared to the unreplicated case.
On the throughput side, we only lose 13\% at a given \gls{slo}.
This shows that our original idea, moving consensus to the transport layer, can be both a general and high-performance tool for the distributed application developer. 

One key finding from this work is that latency of follower nodes matter a lot.
This means that application code should run on a separate thread on followers, to ensure the network code is not waiting on slow application requests.
We observed a 2x throughput gain when implementing this separation of threads.

We believe that moving the consensus to the transport layer is the way to go for easy development of distributed applications.
However, before this solution can be used, there are some challenging issues to solve.
In particular, log compaction and persistent storage must be implemented to ensure true reliable consensus.
This could be a good extension to this project for a future master thesis.

\section*{Acknowledgements}

I would like to thank Marios Kogias for his help during all phases of this project.
Thank you also to the whole DCSL/VLSC team for the interesting lunch time discussion and teachings.
