\chapter{Evaluation}
\label{chap:evaluation}

In this chapter, we measure the performance of our implementation.
In particular, we will focus on two key characteristics: latency and throughput.
Throughput is defined as the number of messages than can be succesfully replicated per second.
Latency is the time between the sending of a request and the reception of the reply (measured at the client).

We designed three different experiments to explore how our system reacts to different conditions.
In the first experiment, we measure the latency of a replicated operation under different fixed throughputs.
This will allow us to measure the peak throughput of the system, defined as the point at which the latency increases infinitely because requests are queued.
This experiment was carried using a cluster of three nodes.

In the second exeperiment, we measured the impact of cluster size on latency.
Since the leader must wait for a bigger quorum of machines to acknowledge the request, we expect the latency to go up with the number of machines.
We also include the case where there is only one machine in the group, i.e. when non replicated requests are used.
This was conducted at a fixed throughput.

For the last experiment, we wanted to measure the impact of leader failure on the system's throughput.
To do so, we modified the code of the server program to fall back to being a follower after a given number of messages was sent.
A new leader election would then take place and the requests sent while the cluster was leaderless would be lost.

All experiments were done on machines equipped with two Intel Xeon E5-2650 CPUs running at \SI{2.6}{\giga\hertz}.
Each machine had \SI{62}{\giga\byte} of RAM, out of which XXX were used for the application. % TODO: How many RAM for the app?
The machines were equipped with Intel~82599ES \glspl{nic} connected to a \SI{10}{\giga\bit\per\second} switch.
All the server machines were running the kernel bypass implementation of the stack, while the client was running the userland one.

\section{Latency - Throughput}

replicated vs non replicated

\section{Impact of cluster size}

at fixed load

\section{Latency over time}

including leader recovery

\section{Comparison with Kernel Paxos}

Kernel Paxos\cite{kernelpaxos} is an earlier attempts to reduce consensus protocol latency by removing the cost of context switching.
In order to do so, they implemented the Paxos protocol as a set of Linux kernel modules.
They had to port the libpaxos implementation, which was intended to run in userland, to things like kernel memory allocation.
Here are the main differences between their approach and ours:

\begin{itemize}
    \item Kernel Paxos is implemented as a set of Linux kernel modules.
        Our implementation uses Intel's DPDK\cite{dpdk} to access the \gls{nic} from userland.
    \item Kernel Paxos uses the Paxos algorithm.
        Our solution uses Raft.
    \item Kernel Paxos uses Ethernet frames to send its messages.
        This requires all machines to be in the same broadcast domain.
        Our implementation runs on top of UDP/IP and therefore can be routed.
\end{itemize}

The Kernel Paxos source code is freely available on the internet\footnote{\url{https://github.com/esposem/Kernel_Paxos}}.
Unfortunately we were not able to run it on our experimental setup.
Therefore, direct performance comparison might not be very accurate.
In particular, their node's CPU is much older than ours (2008 vs 2012).
