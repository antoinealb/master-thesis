\chapter{Evaluation}
\label{chap:evaluation}

In this chapter, we measure the performance of our implementation.
In particular, we will focus on two key characteristics: latency and throughput.
Throughput is defined as the number of requests per second processed by the application.
Latency is the time between the sending of a request and the reception of the reply (measured at the client).
To generate the load and measure the latency, the Lancet distributed load generator\cite{lancet} was used.

We designed two different experiments to explore how our system reacts to different conditions.
In the first experiment, we measure the latency as a function of the achieved throughput.
This will allow us to measure the peak throughput of the system, defined as the point at which the latency increases infinitely because requests are queued.
This experiment was carried using a cluster of three nodes.

In the second experiment, we measured the impact of cluster size on latency.
Since the leader must wait for a bigger quorum of machines to acknowledge the request, we expect the latency to go up with the number of machines.
We also include the case where there is only one machine in the group, \ie when non replicated requests are used.

All experiments were done on machines equipped with two Intel Xeon E5-2650 CPUs running at \SI{2.6}{\giga\hertz}.
Each machine had \SI{64}{\giga\byte} of RAM, out of which \SI{2}{\giga\byte} were allocated to DPDK.
The machines were equipped with Intel~82599ES \SI{10}{\giga\bit\per\second} \glspl{nic} connected through a switch.
All the server machines were running the kernel bypass implementation of the stack, while the client was running the userland one.

\section{Latency - Throughput}
\label{sec:latency_throughput}

In this experiment, we varied the load on a three machine cluster cluster and measured the 99th percentile latency of the requests.
The idea here is that the latency graph should exhibit a large vertical asymptote as we reach the maximum capacity of the system. 
To have a baseline performance, we also ran the same load test on a non-replicated cluster.


\begin{figure}[p]
    \centering
    \begin{subfigure}[t]{0.8\textwidth}
        \includegraphics[width=\textwidth]{plots/latency_10us_synthetic.pdf}
        \caption{\SI{10}{\micro\second} synthetic workload}
    \end{subfigure}

    \begin{subfigure}[t]{0.8\textwidth}
        \includegraphics[width=\textwidth]{plots/latency_1us_synthetic.pdf}
        \caption{\SI{1}{\micro\second} synthetic workload}
    \end{subfigure}
    \caption{
        Throughput of the system both with followers executing application code both in the network thread and in a separate thread. 
        \label{fig:latency-throughput}
    }
\end{figure}

Figure~\ref{fig:latency-throughput} shows that the latency of replica acknowledgment matters a lot for this type of system.
Running the synthetic workload in the network thread (in a symmetric design) has very bad performance.
For example, at a \SI{200}{\micro\second} latency \gls{slo}, running the workload in a separate thread has twice the throughput of the same-thread implementation.
This was not what we expected, and had to be retrofitted in the implementation.
This discovery should definitely be taken in account when designing future consensus systems.

\section{Impact of cluster size}

% TODO: Maybe redo this experiment

In this experiment, the goal was to see how the number of replicas impacted system performance.
Adding new machines to the system increases its reliablity by increasing the number of failures required to have a loss of consistency.
However, we can expect a loss of system performance when gaining in reliability.
Throughput will go down, as adding replicas will increase the number of messages per request.
It should also increase tail latency, but not by much, as we must wait on a bigger quorum of machines.
The results can be see on Figure~\ref{fig:cluster-size}.

\begin{figure}[p]
    \centering

    \begin{subfigure}[t]{0.8\textwidth}
        \includegraphics[width=\textwidth]{plots/cluster_size_10us}
        \caption{\SI{10}{\micro\second} synthetic workload}
    \end{subfigure}

    \begin{subfigure}[t]{0.8\textwidth}
        \includegraphics[width=\textwidth]{plots/cluster_size_1us}
        \caption{\SI{1}{\micro\second} synthetic workload}
    \end{subfigure}

    \caption{Impact of quorum size on system performance.
        This experiment was performed using the worker thread implementation.
    \label{fig:cluster-size}
    }
\end{figure}


% \section{Impact of leader crash}
% 
% % TODO: Do this experiment
% 
% Since in Raft all the requests go through the leader, we wanted to evaluate the impact of a leader crash on throughput.
% Our system does not queue requests in a situation with no leaders; they are simply dropped.
% Therefore we expect the throughput to drop to zero during leader re-election, then to rise back to normal operating levels.
% 
% The results can be seen on Figure~\ref{fig:leader-crash}.
% Note that the duration of the re-election period depends on the chosen values for the Raft timers.
% In this situation, the heartbeat period is set to \SI{10}{\milli\second}, and the election timer period is uniformly distributed between \SIrange{500}{1000}{\milli\second}.
% As they could be tuned to other values, this is more of a qualitative result.


\section{Comparison with Kernel Paxos}

Kernel Paxos\cite{kernelpaxos} is an earlier attempts to reduce consensus protocol latency by removing the cost of context switching.
In order to do so, they implemented the Paxos protocol as a set of Linux kernel modules.
They had to port the libpaxos implementation, which was intended to run in userland, to things like kernel memory allocation.
Here are the main differences between their approach and ours:

\begin{itemize}
    \item Kernel Paxos is implemented as a set of Linux kernel modules.
        Our implementation uses Intel's DPDK\cite{dpdk} to access the \gls{nic} from userland.
    \item Kernel Paxos uses the Paxos algorithm.
        Our solution uses Raft.
    \item Kernel Paxos uses raw Ethernet frames to send its messages.
        This requires all machines to be in the same broadcast domain.
        Our implementation runs on top of UDP/IP and therefore can be routed.
\end{itemize}

The Kernel Paxos source code is freely available on the internet\footnote{\url{https://github.com/esposem/Kernel_Paxos}}.
Unfortunately we were not able to run it on our experimental setup.
Therefore, direct performance comparison might not be very accurate.
In particular, their node's CPU is much older than ours (2008 vs 2012).

The most interesting result is latency: Kernel Paxos claims to have a median latency of \SI{52}{\micro\second}.
We compare this with our \SI{1}{\micro\second} synthetic workload, which is closer to the Kernel Paxos case (which does not do any computation on the chosen values).
We achieve a tail latency of \SI{31}{\micro\second}, despite a Paxos optimization called Fast Paxos\cite{lamport2006fast}.
In Fast Paxos, clients can bypass the leader, saving one network hop.

While Fast Paxos could be theoretically applied to our system, we decided not to for several reasons.
First, we already had a working Raft implementation at this point in our work, and Fast Paxos is significantly harder to implement than Paxos.
Then, Fast Paxos requires more machines to reach consensus: $2/3$ of the nodes must be healthy for a value to be accepted.
This means that to handle $n$ failures, Fast Paxos requires $3n + 1$ replicas, where classic Paxos and Raft only require $2n + 1$.
Note that if the $3n + 1$ replica conditions is not met, Fast Paxos will fall back to classic Paxos.
Finally, under high loads, the probability of message collision is quite high, and in case of collisions the number of network hops in Fast Paxos is the same as with Classic Paxos or Raft.

On the throughput side, Kernel Paxos claims \SI{175}{\kilo msg\per\second}, whereas our implementation reached \SI{275}{\kilo msg\per\second} in a three machines cluster, a $1.6x$ increase.
According to Kernel Paxos authors, their throughput is bound by CPU performance on the proposers.
We were not able to verify this claim, as we were not able to run their implementation in our cluster.
As they did not publish the model of their CPU for their \SI{10}{\giga\bit\per\second} experiments, a direct comparison is hard to do.

In terms of implementation complexity, Kernel Paxos is about 8800 lines of code.
Our implementation is about half this size at 4300 lines.
In order to keep the comparison meaningful, the userland UDP/IP stack is not accounted in those counts.
