\chapter{Background}

\section{Distributed Consensus}

In recent years, datasets have become larger and larger.
Internet-scale companies started using horizontal scaling, in which a large number of inexpensive machines is used.
This approach allowed them to keep up with the increase of sets, but also introduced challenges regarding data consistency.
For example, in a distributed database, some nodes could see transaction A before transaction B, while other nodes could see them in a different order.
If only one of the transaction could proceed, this would create an issue.

Horizontal scaling also created issues of \emph{availability}.
The increased number of servers meant an increase in the probabiltiy of a node failure.

Distributed consensus is the problem of agreeing on a shared state in face of unreliable communications.
It is understandable then that once solution to the distributed consensus problems were available, they quickly became parts of almose every modern large scale system.

why does it matter?

\begin{enumerate}
    \item datasets become bigger 
    \item horizontal scaling using commodity hardware is more convenient and cheaper than large mainframes.
    \item however the parallelism it creates means new issues arise.
    \item especially: how do you ensure consistency of data?
        Maybe one node executes A then B while the other one does B then A.
    \item Also creates issues of fault tolerance; many systems, each with relatively low individual resilience.
    \item The problem of agreeing ona value is called ``distributed consensus problem''
\end{enumerate}

Maybe something about state machine replication?

